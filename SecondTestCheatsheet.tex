\documentclass[9pt]{article}

\usepackage{sectsty}
\usepackage{graphicx}
\usepackage{amsmath}

% Margins
\topmargin=-0.45in
\evensidemargin=0in
\oddsidemargin=0in
\textwidth=6.5in
\textheight=9.0in
\headsep=0.25in

\title{ PD2 Formulu lapa}
\author{ Jorens Štekeļs}
\date{\today}

\begin{document}

\section{Patēriņa teorija}

\begin{align*}
    I &= P_x  \cross Q_x + P_y \cross Q_y
\end{align*}

\begin{itemize}
    \item $I$ – patērētāja budžets 
    \item $P_x$ – preces $x$ cena 
    \item $P_y$ – preces $y$ cena 
    \item $O_x$ – preces $x$ daudzums 
    \item $Q_y$ – preces $y$ daudzums 
\end{itemize}

\begin{align*}
    \Delta I_{real} &= \Delta I_{nominal} - \Delta P_{level\%}
\end{align*}

\section{Ražošana}

\begin{align*}
	AP &= TP:L \\
	MP &= \Delta TP:\Delta L \\
	TC &= FC + VC  \\
	ATC &= \frac{VC+FC}{Q} \\
	AVC &= \frac{VC}{Q} \\
	AFC &= \frac{FC}{Q} \\
	MC &= \Delta TC:\Delta Q  
\end{align*} 
\begin{itemize}
    % TK - mby expand
    \item AP - vidējais produkts,
    \item MP - galējais produkts,
    \item ATC - vid. kop. produkts,
    \item AVC - vid. mainīgais produkts,
    \item AFC - vidējais fiksētais products,
    \item L - laiks.
\end{itemize}

\section{Konkurence}

\begin{align*}
    MR &= \frac{\Delta TR}{\Delta Q}
\end{align*}

\end{document}
